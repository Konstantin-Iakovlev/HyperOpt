\documentclass[aspectratio=169]{beamer}
\beamertemplatenavigationsymbolsempty
\usecolortheme{beaver}
\setbeamertemplate{blocks}[rounded=true, shadow=true]
\setbeamertemplate{footline}[page number]


\usepackage{graphicx}
\usepackage{amsmath}
\usepackage{amsfonts}
\usepackage{caption}
\usepackage{subcaption}
\usepackage{float}
\usepackage[english,russian]{babel}
\usepackage{xcolor}



\usepackage[utf8]{inputenc}
\usepackage{amssymb,amsfonts,amsmath,mathtext}
%\usepackage{subfig}
\usepackage{tikz}
\usepackage{xcolor}
\usepackage[all]{xy} % xy package for diagrams
\usepackage{array}
\usepackage{multicol}% many columns in slide
\usepackage{hyperref}% urls
\usepackage{hhline}%tables


\def\bw{\mathbf{w}}
\def\balpha{\boldsymbol{\alpha}}

\definecolor{dark_green}{rgb}{0, 0.788, 0}
\definecolor{dark_red}{rgb}{0.9, 0, 0}


% Your figures are here:
\graphicspath{ {fig/} {../fig/} }

\definecolor{ao(english)}{rgb}{0.0, 0.5, 0.0}
\definecolor{bleudefrance}{rgb}{0.19, 0.55, 0.91}

%----------------------------------------------------------------------------------------------------------
\title[\hbox to 56mm{Feature generation}]{Поиск согласованных нейросетевых моделей в задаче мультидоменного обучения}
\author{К.\,Д.~Яковлев\inst{1} \and \and О.\,Ю.~Бахтеев\inst{1,2}\and В.\,В.~Стрижов\inst{1,2} \\
\tt{\footnotesize \{iakovlev.kd, bakhteev, strijov\}@phystech.edu }}
\institute{\inst{1} Москва, Московский физико-технический институт \and
\inst{2} Москва, Вычислительный центр им. А.А. Дородницына ФИЦ ИУ РАН} \date{2023}
%----------------------------------------------------------------------------------------------------------
\begin{document}
%----------------------------------------------------------------------------------------------------------
\begin{frame}
\thispagestyle{empty}
\maketitle
\end{frame}
%-----------------------------------------------------------------------------------------------------
\begin{frame}{Цель исследования}

\begin{block}{Цель} 
  Предложить градиентный метод оптимизации гиперпараметров с линейейной по количеству параметров и гиперпараметров
  сложностью итерации и затратами памяти.
\end{block}

~\\
\begin{block}{Проблема}
  Существующие методы не гарантируют выполнение следующих условий одновременно: 1) онлайн оптимизация,
  2) отсутствие смещения из-за короткого горизонта, 3) линейная сложность итерации и затраты памяти.
\end{block}
~\\
\begin{block}{Метод решения}
  Предлагаемый метод основан на аггрегации жадных гиперградиентов без дополнительных вычислительных затрат.
\end{block}

\end{frame}


\begin{frame}{Аггрегация жадных гиперградиентов}
  Пусть задано $\gamma \in (0, 1)$. Тогда аппроксимация гиперградиента запишется как:
  \begin{align*}
    \hat{\nabla}_{\boldsymbol{\alpha}} = \frac{\partial}{\partial \boldsymbol{\alpha}}\mathcal{L}_2(\mathbf{w}_T(\boldsymbol{\alpha}), \boldsymbol{\alpha}) +
    \sum_{t=1}^T \mathbf{B}_t\frac{\partial\mathcal{L}_2(\mathbf{w}_t, \boldsymbol{\alpha})}{\partial \mathbf{w}_t}\gamma^{T-t}.
  \end{align*}
  \begin{table}
    \begin{tabular}{c|c|c|c|c|c}
       & IFT & RMAD & DrMAD & TruncBP & Proposed \\
      Онлайн оптимизация & \color{dark_red}{\texttimes} & \color{dark_green}{\checkmark} & \color{dark_red}{\texttimes} & \color{dark_green}{\checkmark} & \color{dark_green}{\checkmark} \\
      Длинный горизонт & \color{dark_green}{\checkmark} & \color{dark_green}{\checkmark} & \color{dark_green}{\checkmark} & \color{dark_red}{\texttimes} & \color{dark_green}{\checkmark} \\
      линейная сложность & \color{dark_green}{\checkmark} & \color{dark_red}{\texttimes} & \color{dark_green}{\checkmark} & \color{dark_green}{\checkmark} & \color{dark_green}{\checkmark} \\
      
    \end{tabular}
  \end{table}
  
\end{frame}

%----------------------------------------------------------------------------------------------------------


% \begin{frame}{Основная литература}
% \begin{thebibliography}{1}


% \bibitem{darts} 
% Hanxiao Liu and Karen Simonyan and Yiming Yang. 
% \textit{DARTS: Differentiable Architecture Search}.
% CoRR, 2018.


% \bibitem{wang2021multi}
% Wang, Q., Ke, J., Greaves, J., Chu, G.,
% Bender, G., Sbaiz, L., Go, A., Howard, A., Yang, M.,
% Gilbert, J. \& Others
% \textit{Multi-path neural networks for on-device multi-domain visual classification}.
% CoRR, 2021.

% \bibitem{darts-cc}
% Yakovlev, K., Grebenkova, O., Bakhteev, O. \& Strijov, V.
% \textit{Neural Architecture Search with Structure Complexity Control}.
% CoRR, 2022.

% \end{thebibliography}	
% \end{frame}

\begin{frame}{Постановка задачи оптимизации гиперпараметров}
\begin{itemize}
  \item Пусть задан вектор параметров модели $\mathbf{w} \in \mathbb{R}^p$ и вектор гиперпараметров
  $\boldsymbol{\alpha} \in \mathbb{R}^h$. задача оптимизации:
  \begin{align*}
    \boldsymbol{\alpha}^* &= \arg\min_{\boldsymbol{\alpha}}\mathcal{L}_2(\mathbf{w}^*, \boldsymbol{\alpha}), \\
    \mathrm{s.t.} \quad \mathbf{w}^* &= \arg\min_{\mathbf{w}}\mathcal{L}_1(\mathbf{w}, \boldsymbol{\alpha}).
  \end{align*}
  \item Пусть внутренняя задача решается с помощью оптимизатора $\mathbf{\Phi}(., .)$:
  \begin{align*}
    \mathbf{w}_{t + 1}(\boldsymbol{\alpha}) = \mathbf{\Phi}(\mathbf{w}_t, \boldsymbol{\alpha}), \quad
    t = \overline{1, T}.
  \end{align*}
  \item Гиперградиент запишется как:
  \begin{align*}
    &\nabla_{\boldsymbol{\alpha}}\mathcal{L}_2(\mathbf{w}_T(\boldsymbol{\alpha}), \boldsymbol{\alpha}) = 
    \frac{\partial}{\partial \boldsymbol{\alpha}}\mathcal{L}_2(\mathbf{w}_T(\boldsymbol{\alpha}), \boldsymbol{\alpha}) + 
    \sum_{t=1}^T\mathbf{B}_t\mathbf{A}_{t+1}\ldots\mathbf{A}_T
    \frac{\partial \mathcal{L}_2(\mathbf{w}_T(\boldsymbol{\alpha}), \boldsymbol{\alpha})}{\partial\mathbf{w}}, \\
    &\mathbf{B}_t = \frac{\partial\mathbf{\Phi}(\mathbf{w}_{t-1}, \boldsymbol{\alpha})}{\partial\boldsymbol{\alpha}},
    \quad \mathbf{A}_t = \frac{\partial\mathbf{\Phi}(\mathbf{w}_{t-1}, \boldsymbol{\alpha})}{\partial\boldsymbol{w}_{t - 1}}.
  \end{align*}
\end{itemize}
\end{frame}


\begin{frame}{Аппроксимация гиперградиента}
  \begin{itemize}
    \item Пусть задано $\gamma \in (0, 1)$. Тогда аппроксимация гиперградиента запишется как:
  \begin{align*}
    \hat{\nabla}_{\boldsymbol{\alpha}} = \frac{\partial}{\partial \boldsymbol{\alpha}}\mathcal{L}_2(\mathbf{w}_T(\boldsymbol{\alpha}), \boldsymbol{\alpha}) +
    \sum_{t=1}^T \mathbf{B}_t\frac{\partial\mathcal{L}_2(\mathbf{w}_t, \boldsymbol{\alpha})}{\partial \mathbf{w}_t}\gamma^{T-t}.
  \end{align*}
  \item предлположения:
  \begin{enumerate}
    \item $\mathcal{L}_1(., \boldsymbol{\alpha})$, $\mathcal{L}_2(., \boldsymbol{\alpha})$ являются $L$-гладкими 
    и $\mu$-сильно выпуклыми.
    \item $\frac{\partial^2\mathcal{L}_1(., \boldsymbol{\alpha})}{\partial\mathbf{w}\partial\mathbf{w}^\top}$ является $H_w$-липшицева.
    \item $1 - \eta L \leq \gamma 1 - \eta\mu$
    \item $\|\frac{\partial\mathcal{L}_1(\mathbf{w}, \boldsymbol{\alpha})}{\partial\boldsymbol{\alpha}\partial\mathbf{w}^\top}\| \leq B$.
    \item $\frac{\partial^2\mathcal{L}_1(., \boldsymbol{\alpha})}{\partial\boldsymbol{\alpha}\partial\mathbf{w}^\top}$ является $M_b$-липшицевой.
    \item $(\frac{\partial^2\mathcal{L}_1(., \boldsymbol{\alpha})}{\partial\boldsymbol{\alpha}\partial\mathbf{w}^\top})^\top
    (\frac{\partial^2\mathcal{L}_1(., \boldsymbol{\alpha})}{\partial\boldsymbol{\alpha}\partial\mathbf{w}^\top}) \succeq \kappa \mathbf{I}$.
  \end{enumerate}
  \end{itemize}
\end{frame}


\begin{frame}{Ассимптотическая несмещенность гиперградиента}

  \begin{block}{Теорема (Яковлев, 2023)}
    Пусть выполнены предположения (1-6). Тогда:
    \begin{align*}
     &\|\hat{\nabla}_{\boldsymbol{\alpha}} - \nabla_{\boldsymbol{\alpha}}\|_2 \leq \frac{2LB\|\mathbf{w}_0 - \mathbf{w}_*\|\sqrt{1 - \eta\mu}^T}{\sqrt{1 - \eta\mu}^{-1} - 1} +
     B\|\frac{\partial\mathcal{L}_2(\mathbf{w}_T, \boldsymbol{\alpha})}{\partial\mathbf{w}}\| \cdot \\
     &\left[
      \frac{1}{\eta}(\frac{1}{\mu} - \frac{1}{L} + \frac{1}{L}(1 - \eta\mu)^T) + 2\eta H_w((T - 1)\sqrt{1 - \eta\mu}^T- 
      \frac{\sqrt{1 - \eta\mu}^{T-1} - (1 - \eta\mu)^T}{\sqrt{1 - \eta\mu}^{-1} - 1})
     \right].
    \end{align*}
  \end{block}

  \begin{block}{Теорема (Яковлев, 2023)}
    Пусть $\mathcal{L}_2 = \mathcal{L}_2(\bw)$. Тогда найдется $c > 0$:
    \begin{align*}
      \hat{\nabla}_{\boldsymbol{\alpha}}^\top\nabla_{\boldsymbol{\alpha}} \geq c\|\nabla_{\bw}\mathcal{L}_2(\bw_T, \boldsymbol{\alpha})\|_2^2.
    \end{align*}
    То есть, выполнено достаточное условие спуска.
  \end{block}
  
\end{frame}



\begin{frame}{Постановка вычислительного эксперимента}
  \begin{itemize}
    \item Цель -- сравнение качества предложенного подхода с существующими методами подсчета гиперградиента.
    \item Эксперимент проводится на задаче очистки обучающей выборки. Приводится точность предсказания на отложенной выборке.
    \item Сравниваются следующие методы: DrMAD, IFT, Truncated Backpropagation.
  \end{itemize}

  \begin{table}[h]
    \centering
    \begin{tabular}{c|c|c}
        \textbf{Method} & \textbf{Valid. Acc.} & $\#$\textbf{JVPs} \\ \hline
        Truncated backpropagation (Lukethina) & 72.5 & $1 \; (1)$ \\
        DrMAD & 69.8 & $99 \; (2T - 1)$ \\
        IFT(9, 5) & 70.3 & $50 \;  ((N + 1)K)$ \\
        IFT(4, 10) & 70.7 & $50 \; ((N + 1)K)$ \\
        Proposed ($\gamma = 0.99$) & $\textbf{73.5}^*$ & $50 \; (T)$ \\ \hline
    \end{tabular}
\end{table}
Из таблицы видно, что предложенный метод превосходит существующие методы оптимизации гиперпараметров в терминах 
точности предсказаний на отложенной выборке, имея сопоставимые вычислительные затраты.
\end{frame}






% \begin{frame}{Постановка вычислительного эксперимента}

% \begin{itemize}
% \item Цель -- получение зависимости качества работы мультимодели и
% количества ее параметров в зависимости от используемого регуляризатора.
% \item Эксперимент проводится на подвыборке MNIST. В качестве доменов рассматриваются
% изображения, повернутые на угол, кратный $\pi/2$. Число доменов меняется от 1 до 4.
%  Сравниваются следующие модели:
% мультимодель со структурной регуляризацией, мультимодель с регуляризацией скрытых представлений,
% а также модель, не учитывающая разбиение выборки на домены.
% \item Оценивается средняя точность (accuracy) на тестовой выборке для каждого из доменов.
% Также приводится количество параметров для каждой модели.
% \end{itemize}

% \end{frame}


% \begin{frame}{Результаты вычислительного эксперимента}
%   \begin{table}[h!]
%     \centering
%      \begin{tabular}{||c c c||}
%      \hline
%      model & accuracy & num. of params \\ [0.5ex] \hline\hline
%      \multicolumn{3}{|| c ||}{1 domain} \\
%       single & 60.59 & 5029 \\ \hline \hline 

%      \multicolumn{3}{|| c ||}{2 domains} \\
%      single, union & 66.95 & 6560 \\
%      multimodel, struct & 62.86 & \textbf{5248} \\
%      multimodel, contr & \textbf{69.64} & 9328 \\
%      \hline \hline

%      \multicolumn{3}{|| c ||}{3 domains} \\
%      single, union & 63.02 & 5685\\
%      multimodel, struct & 64.85 & \textbf{6826}\\
%      multimodel, contr & \textbf{65.01} & 12096 \\
%      \hline\hline

%      \multicolumn{3}{|| c ||}{4 domains} \\
%      single, union & 67.16 & 6560\\
%      multimodel, struct & 63.15 & \textbf{7872}\\
%      multimodel, contr & \textbf{67.98} & 13685 \\
%      \hline\hline
%      \end{tabular}
%     \end{table}
% \end{frame}


% \begin{frame}{Вычислтельный эксперимент на задче языкового моделирования}
%   \begin{enumerate}
%     \item Задача языкового моделирования решалась на выборке IWSLT14.
%     \item Расматривались одиночные модели, решающие задачу на одном домене и на
%     объединении доменов, а также предлагаемая мультимодель.
%   \end{enumerate}
%   \begin{table}
%     \centering
%     \begin{tabular}{|c|c|c|}
%       \hline
%       \textbf{Model} & \textbf{En, ppl} & \textbf{De, ppl} \\ \hline
%       Single, En & \textbf{59.35} & - \\ 
%       Single, De & - & - \\ \hline
%       Single, En+De & 72.68 & 82.15 \\
%       Multimodel & \textbf{64.91} & \textbf{74.06} \\ \hline
%     \end{tabular}
%   \end{table} 
% \end{frame}



\begin{frame}{Заключение}
    % \begin{itemize}
    % \item Рассмотрена задача поиска архитектуры модели глубокого
    % обучения на мультидоменных данных. Задача рассматривалась как
    % задача мультимоделирования.
    % \item Предложены два метода регуляризации: регуляризация структуры и регуляризация
    % пространства скрытых представлений модели.
    % \item Продемонстрирована работоспособность предлагаемого решения. При использовании
    % первого регуляризатора мультимодель имеет меньшее число параметров. При использовании
    % второго регуляризатора модель имеет лучшую точность классификации.
    % \item В дальнейшем планируется провести вычислительный эксперимент на задаче
    % мультиязычного языкового моделирования.
    % \end{itemize}
    \begin{itemize}
      \item Рассмотрена задача оптимизации гиперпараметров.
      \item Предложен метод оптимизации гиперпараметров, удовлетворяющий одновременно трем условиям:
      1) онлайн оптимизация, 2) отсутствие смещения из-за короткого горизонта, 3) линейная сложность итерации и затраты памяти.
      \item Продемонстрирована работоспособность предлагаемого решения.
    \end{itemize}
\end{frame}

%----------------------------------------------------------------------------------------------------------
\end{document} 